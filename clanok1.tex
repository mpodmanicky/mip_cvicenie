% Metódy inžinierskej práce

\documentclass[10pt,twoside,a4paper]{article}
\usepackage[]{babel}
%\usepackage[T1]{fontenc}
\usepackage[]{fontenc} % lepšia sadzba písmena Ľ než v T1
\usepackage[utf8]{inputenc}
\usepackage{graphicx}
\usepackage{url} % príkaz \url na formátovanie URL
\usepackage{hyperref} % odkazy v texte budú aktívne (pri niektorých triedach dokumentov spôsobuje posun textu)


\usepackage{cite}
%\usepackage{times}

\pagestyle{headings}

\title{Rozšírená realita a jej využitie v každodennom živote\thanks{Semestrálny projekt v predmete Metódy inžinierskej práce, ak. rok 2022/23, vedenie: Igor Stupavský}} % meno a priezvisko vyučujúceho na cvičeniach

\author{Martin Podmanický\\[2pt]
	{\small Slovenská technická univerzita v Bratislave}\\
	{\small Fakulta informatiky a informačných technológií}\\
	{\small \texttt{xpodmanickym@stuba.sk}}
	}

\date{\small 18. október 2022} % upravte



\begin{document}

\maketitle

\begin{abstract}
Vybraná téma rozšírenej reality a ako nám, ľuďom, môže dopomôcť k jednoduchšiemu a rýchlejšiemu riešenia problémov v každodennom živote.
V profesionálnom živote to má tiež svoje výhody, ku ktorým sa dostanem v priebehu článku. 
Ten venujem čitateľom ktorý sa chcú dozvedieť niečo viac o rozšírenej realite a jej fungovaní.
Taktiež, poskytnem vlastné nápady využitia, implementovania do skutočného života a efektivitu využitia.
\ldots
\end{abstract}



\section{Úvod}

Touto prácou by som chcel priblížiť fungovanie a výhody rozšírenej reality. V našej pokrokovej dobe sú mnohé výhody techniky a elektroniky.
čo tak teda keby sme spojili náš svet a svet techniky do jedného a žili v symbióze. Keďže vieme, že počítač bude vždy dominovať v efektivite a riešení problémov nad človekom,
som za nápad spolunažívania.


Priblíženie a fungovanie Rozšírenej reality~\ref{co} a~\ref{ako}.
Jej využitie v momentálnej dobe~\ref{vyuzitie}.
Využitie v iných odvetviach~\ref{zdravotnictvo}.
Virtuálna realita a vážne hry~\ref{vz}.
Zhodnotenie a spätný pohľad prináša sekcia~\ref{konec}.
Záverečné poznámky prináša časť~\ref{zaver}.



\section{Čo to vlastne je} \label{co}

Rozšírená realita nie je nič zložité ani náročné na pochopenie. Ide v podstate o to, čomu názov nasvedčuje = rozširuje realitu. Rozširuje realitu o prvky, ktoré v skutočnosti nie sú fyzicky hmatateľné. Vrstva z digitálneho prostredia je tak implementovaná do našich životov a narozdiel od virtuálnej reality, ktorá vytvára aj umelé prostredie, nijak neprekáža v normálnom fungovaní života ba podľa všetkého by k tomu mala ešte dopomáhať a zefektívniť, aspoň teda vybrané zariadenia. V súčastnosti rozšírená realita vytvára  3D koncept, ktorý sa priamo premieta do prostredia. Najväčšie využitie je však v zábavnom priemysle.

\section{Ako to funguje}\label{ako}

Teraz keď už máme obraz o tom, čo to je, poďme zistiť ako to celé funguje. Používajú sa rôzne obsahy, konkrétnejšie obrázky, animácie, 3D modely alebo aj video, ktoré môžu byť zobrazované na iných zariadeniach  s obrazovkou. Mobilný telefón, displej a podobne. K fungovaniu rozšírenej reality, je za potreba senzorov. Cez fotoaparát a iné senzory na zariadení sa zhromažďujú informácie o užívateľovom okolí a jeho interakciách. Tieto informácie sa následne odosielajú na spracovanie. Nakoniec, na naskenované prostredie cez fotoaparát alebo kameru v zariadení, je možné umiestňovať do prostredia vygenerované modely. Práve toto bol obrovský úspech hry PokemonGO, ktorá využívala systém rozšírenej reality. K rozšírenej realite patrí však aj projekcia. Tak ako pracuje s prostredím a sníma okolie, dokáže doň cez projektory vkladať generované 3D modely. Predstavte si to ako hologramy z filmov ale také, s ktorými je možno interakcie.

Z obr.~\ref{f:rozhod} je všetko jasné. 

\begin{figure*}[tbh]
\centering
%\includegraphics[scale=1.0]{diagram.pdf}
Aj text môže byť prezentovaný ako obrázok. Stane sa z neho označený plávajúci objekt. Po vytvorení diagramu zrušte znak \texttt{\%} pred príkazom \verb|\includegraphics| označte tento riadok ako komentár (tiež pomocou znaku \texttt{\%}).
\caption{Rozhodujúci argument.}
\label{f:rozhod}
\end{figure*}



\section{Ako ju vie využiť naša spoločnosť} \label{vyuzitie}
\subsection{Súčasnosť}\label{vyuzitie:suc}
V súčasnej dobe sa stretávame s rozšírenou realitou najmä na našich smartfónoch. Najznámejšie budú filtre z aplikácie Snapchat, ktoré nasnímajú vašu tvár a potom podľa vašej preferencie na ňu premietne model alebo obraz. Takisto v už aj spomínanej hre PokemonGO\ref{2} v úvode. Ďalej používajú rozšírenú realitu firmy, ktoré si tak zlepšujú marketing. Porsche umožňuje užívateľovi náhľad ako by to vyzeralo keby má pri svojom dome auto ich značky. Taký istý princíp využívajú aj stavebné firmy, obchodné reťazce s oblečením a podobne.


Môže sa zdať, že problém vlastne nejestvuje\cite{}, ale bolo dokázané, že to tak nie je~\cite{}. Napriek tomu, aj dnes na webe narazíme na všelijaké pochybné názory\cite{}. Dôležité veci možno \emph{zdôrazniť kurzívou}.


\subsection{Budúcnosť} \label{vyuzitie:bud}

Momentálne je koncept rozšírenej reality najčastejšie využívaný pre zábavné účely. Avšak mnoho firiem ako Apple a Google pracujú na okuliaroch s rozšírenou realitou. Taktiež Mojo Vision vytvára projekt kontaktných šošoviek, ktoré využívajú microLED displej, slúžiaci na zobrazovanie kritických informácií, a sú taktiež vybavené smart senzormi ktoré korigujú zrak. Ich cieľom je implementovanie rozšírenej reality aby sme my ako ľudia mohli byť najlepšou verziou seba.~\cite{Mojo} Ďalšie využitie by si našla vo vyučovaní. Pomohlo by to najmä tým, ktorí sú takzvaní "visual-learners" čiže sa učia podľa toho čo vidia. Rozšírená realita by mohla pomôcť takýmto študentom lepšie pochopiť a viac experimentovať s rôznymi prvkami ich výučby.


Niekedy treba uviesť zoznam:

\begin{itemize}
\item jedna vec
\item druhá vec
	\begin{itemize}
	\item x
	\item y
	\end{itemize}
\end{itemize}

Ten istý zoznam, len číslovaný:

\begin{enumerate}
\item jedna vec
\item druhá vec
	\begin{enumerate}
	\item x
	\item y
	\end{enumerate}
\end{enumerate}


\section{Využitie virtuálnej reality v zdravotníctve} \label{zdravotnictvo}
Nástupom 21. storočia sa začali využívať robotom-asistované systémy v zdravotníctve, a tak je len jasné, že využitie virtuálnej reality je len otázka času...alebo lepšie povedané bola. Doposiaľ najpopulárnejší robotický systém v zdravotníctve je da Vinci XI, čo je systém pre robotickú chirurgiu. Chirurgom umožňuje tak vykonávať aj tie najzložitejšie scenáre, keď príde k chirurgií. Tak ako aj Dr. Love hovorí, pacienti podstupujú oveľa nižšie riziko ako pri bežnej operácií a tiež tak dokážu minimalizovať bolesti a post-operačné problémy, tak ako aj skrátiť čas na zotavenie.~\cite{XiRobots}
Prístroj da Vinci XI využíva virtuálnu realitu tak, že dokáže operovanú plochu priblížiť akoby lupou, a tak umožniť chirurgovi lepšiu viditeľnosť. Taktiež dokáže zobrazovať dôležité miesta na operovanej časti. A mnohé ďalšie. Virtuálna realita avšak nemá využitie len tu. Takmer všetky takéto systémy pracujú, že vyobrazujú virtuálne prvky do fyzického sveta, s ktorými je následne možno interakcie. 

\paragraph{Hardvérové prvky virtuálnej reality.}
Virtuálna realita pre svoje fungovanie potrebuje hardvér, ktorý dokáže spracovať to, čo už samotný program virtuálnej reality chce zobraziť. Najlepším príkladom je telefón a hra PokemonGO\ref{2}. Ďalej to sú smart-okuliare ako napríklad Xiaomi Smart Glasses. Nie všetky nositeľné prvky vyžadujú ovládače pre narábanie so zobrazovanými prvkami. Pri smart okuliaroch stačia gestá, poťukanie rámu, možno nahovorenie príkazu, ale tie, ktoré potrebujú ovládače pre manipuláciu majú využitie v hrách. Existujú rôzne hedsety, ktoré práve tieto ovládače používajú, avšak jedným z tých, ktorý to nepotrebuje je Microsoft Holo-Lens, čo je veľmi atraktívny hedset, ktorý má široké možnosti využitia, nie však na hry.



\section{Rozšírená realita a vážne hry} \label{vz}




\section{Zhodnotenie a spätný pohľad} \label{konec}




\section{Záver} \label{zaver} % prípadne iný variant názvu



%\acknowledgement{Ak niekomu chcete poďakovať\ldots}


% týmto sa generuje zoznam literatúry z obsahu súboru literatura.bib podľa toho, na čo sa v článku odkazujete
\bibliography{zdroje}
\bibliographystyle{plain} % prípadne alpha, abbrv alebo hociktorý iný
\end{document}
