% Metódy inžinierskej práce

\documentclass[10pt,twoside,slovak,a4paper]{article}

\usepackage[slovak]{babel}
%\usepackage[T1]{fontenc}
\usepackage[IL2]{fontenc} % lepšia sadzba písmena Ľ než v T1
\usepackage[utf8]{inputenc}
\usepackage{graphicx}
\usepackage{url} % príkaz \url na formátovanie URL
\usepackage{hyperref} % odkazy v texte budú aktívne (pri niektorých triedach dokumentov spôsobuje posun textu)

\usepackage{cite}
%\usepackage{times}

\pagestyle{headings}

\title{Názov\thanks{Semestrálny projekt v predmete Metódy inžinierskej práce, ak. rok 2022/23, vedenie: Igor Stupavský}} % meno a priezvisko vyučujúceho na cvičeniach

\author{Martin Podmanický\\[2pt]
	{\small Slovenská technická univerzita v Bratislave}\\
	{\small Fakulta informatiky a informačných technológií}\\
	{\small \texttt{xpodmanickym@stuba.sk}}
	}

\date{\small 18. október 2022} % upravte



\begin{document}

\maketitle

\begin{abstract}
Vybraná téma rozšírenej reality a ako nám, ľuďom, môže dopomôť k jednoduchšiemu a rýchlejšiemu riešenia problémov v každodennom živote.
V profesionálnom živote to má tieź svoje výhody, ku ktorým sa dostanem v priebehu článku. 
Ten venujem čitateľom ktorý sa chcú dozvedieť niečo viac o rozšírenej realite a jej fungovaní.
Taktiež, poskytnem vlastné nápady využitia, implementovania do skutočného života a efektivitu využitia.
\ldots
\end{abstract}



\section{Úvod}

Touto prácou by som chcel pribížiť fungovanie a výhody rozšírenej reality. V našej pokrokovej dobe sú mnohé výhody techniky a elektroniky.
čo tak teda keby sme spojili náš svet a svet techniky do jedného a žili v symbióze. Keďže vieme, že počítač bude vždy dominovať v efektivite a riešení problémov nad človekom,
som za nápad spolunažívania.

Uveďte explicitne štruktúru článku. Tu je nejaký príklad.
Základný problém, ktorý bol naznačený v úvode, je podrobnejšie vysvetlený v časti~\ref{2}.
Dôležité súvislosti sú uvedené v častiach~\ref{dolezita} a~\ref{dolezitejsia}.
Záverečné poznámky prináša časť~\ref{zaver}.



\section{Čo to vlastne je} \label{2}
Rozšírená realita nie je nič zložité ani náročné na pochopenie. Ide v podstate o to, čomu názov nasvedčuje = rozširuje realitu. Rozširuje realitu o prvky, ktoré v skutočnosti nie sú fyzicky hmatateľné. Vrstva z digitálneho prostredia je tak implementovaná do našich životov a narozdiel od virtuálnej reality, ktorá vytvára aj umelé prostredie, nijak neprekáža v normálnom fungovaní života ba podľa všetkého by k tomu mala ešte dopomáhať a zefektívniť, aspoň teda vybrané zariadenia. V súčastnosti rozšírená realita vztvára  3D koncept, ktorý sa priamo premieta do prostredia. Najväčšie využitie je však v zábavnom priemysle.
\section{Ako to funguje}\label{3}
Teraz keď už máme obraz o tom, čo to je, poďme zistiť ako to celé funguje.
Z obr.~\ref{f:rozhod} je všetko jasné. 

\begin{figure*}[tbh]
\centering
%\includegraphics[scale=1.0]{diagram.pdf}
Aj text môže byť prezentovaný ako obrázok. Stane sa z neho označný plávajúci objekt. Po vytvorení diagramu zrušte znak \texttt{\%} pred príkazom \verb|\includegraphics| označte tento riadok ako komentár (tiež pomocou znaku \texttt{\%}).
\caption{Rozhodujúci argument.}
\label{f:rozhod}
\end{figure*}



\section{Iná časť} \label{ina}

Základným problémom je teda\ldots{} Najprv sa pozrieme na nejaké vysvetlenie (časť~\ref{ina:nejake}), a potom na ešte nejaké (časť~\ref{ina:nejake}).\footnote{Niekedy môžete potrebovať aj poznámku pod čiarou.}

Môže sa zdať, že problém vlastne nejestvuje\cite{Coplien:MPD}, ale bolo dokázané, že to tak nie je~\cite{Czarnecki:Staged, Czarnecki:Progress}. Napriek tomu, aj dnes na webe narazíme na všelijaké pochybné názory\cite{PLP-Framework}. Dôležité veci možno \emph{zdôrazniť kurzívou}.


\subsection{Nejaké vysvetlenie} \label{ina:nejake}

Niekedy treba uviesť zoznam:

\begin{itemize}
\item jedna vec
\item druhá vec
	\begin{itemize}
	\item x
	\item y
	\end{itemize}
\end{itemize}

Ten istý zoznam, len číslovaný:

\begin{enumerate}
\item jedna vec
\item druhá vec
	\begin{enumerate}
	\item x
	\item y
	\end{enumerate}
\end{enumerate}


\subsection{Ešte nejaké vysvetlenie} \label{ina:este}

\paragraph{Veľmi dôležitá poznámka.}
Niekedy je potrebné nadpisom označiť odsek. Text pokračuje hneď za nadpisom.



\section{Dôležitá časť} \label{dolezita}




\section{Ešte dôležitejšia časť} \label{dolezitejsia}




\section{Záver} \label{zaver} % prípadne iný variant názvu



%\acknowledgement{Ak niekomu chcete poďakovať\ldots}


% týmto sa generuje zoznam literatúry z obsahu súboru literatura.bib podľa toho, na čo sa v článku odkazujete
\bibliography{literatura}
\bibliographystyle{alpha} % prípadne alpha, abbrv alebo hociktorý iný
\end{document}
